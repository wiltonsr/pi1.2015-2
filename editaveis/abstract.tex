\begin{resumo}[Abstract]
 \begin{otherlanguage*}{english}
   Due to the number of fatalities that head-on collisions on highways cause, the adoption of Intervehicular Communication System for Collision Alerts in Highways (CIAC) is presented as a promising tool to reduce occurrences of this type of accident in Brazilian two-way roads. The purpose of this study was to create a fast and secure way to alert the driver whether the overtaking act is possible or not. Land vehicles,  with the exception of motorcycles and mopeds, may be equipped with this system, which has its operations based on intervehicular communication composed mainly of transponder, GPS, Lidar, Radar, camera and rotation sensor, devices which together execute the transmission, sense of geographical position, speed, direction and vehicle direction as well as the instantaneous interpretation of data received. In this way, it determines which are the vehicles that make up a situation of risk, calculating the time needed for the safe conduct of the overtaking.


   \vspace{\onelineskip}

   \noindent
   \textbf{Key-words}: collision avoidance system, overtaking, highways, CIAC, collision, alerts
 \end{otherlanguage*}
\end{resumo}
