\begin{resumo}
  O projeto visa desenvolver um sistema anti colisão veícular, para evitar acidentes em rodovias de mão dupla
  municipais, estaduais e federais pertencentes ao território brasileiro durante ultrapassagens. Ele deverá
  funcionar desde a intenção até a conclusão da ultrapassagem. O sistema de alertas que é equipado em veículos automotores terrestres, com exceção de motocicletas e ciclomotores, avisa ao motorista se é possível realizar a manobra com segurança.

  Através da comunicação inter veicular composta pelo par transponder e GPS, os quais realizam a transmissão,
  recepção da posição geográfica, velocidade, direção e sentido, assim como a interpretação instantânea dos dados
  recebidos, são determinados os veículos que estão ao redor, possibilitando o cálculo do tempo para que a
ultrapassagem seja considerada segura. Em conjunto a esse par, para os casos de curta distância, há ainda o
auxilio de sensores, os quais identificam objetos a frente do veículo para informar se a ultrapassagem poderá
ser realizada.

  O projeto é desenvolvido com base em pesquisas bibliográficas, estas incluem a pesquisa de soluções semelhantes,
  tecnologias existentes, estatisticas de acidentes, conceitos ciêntifícos,  para fundamentar a necessidade deste
   sistema, que a proposta da resolução seja única, que é tecnicamente possível e economicamente viável de ser
   produzido e disponibilizado para os usuários finais.

  Assim determina-se a visão geral de qual o problema, a necessidade e os objetivos  almejados por este projeto,
  quais serão as tecnologias e ferramentas envolvidas e a justificativa da utilização destas no sistema. 

 \vspace{\onelineskip}

 \noindent
 \textbf{Palavras-chaves}: \imprimirpalavrachaveum, \imprimirpalavrachavedois, \imprimirpalavrachavetres, \imprimirpalavrachavequatro
\end{resumo}
