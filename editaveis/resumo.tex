\begin{resumo}
Devido ao grande número de vítimas fatais que colisões frontais em rodovias causam, a adoção do Sistema de Comunicação Interveicular para Alertas de Colisões em Rodovias (CIAC) apresenta-se como uma ferramenta promissora para a diminuição de ocorrências desse tipo de acidente em rodovias brasileiras de mão dupla. O propósito desse estudo foi criar uma maneira rápida e segura de alertar ao condutor do veículo se a ultrapassagem é ou não viável. Veículos automotores terrestres, com exceção de motocicletas e ciclomotores, poderão ser equipados com esse sistema, o qual tem seu funcionamento baseado na comunicação interveicular composta principalmente por transponder, GPS, Lidar, Radar, câmera e sensor de rotação, dispositivos que em conjunto realizam a transmissão, percepção da posição geográfica, velocidade, direção e sentido dos automóveis, assim como a interpretação instantânea dos dados recebidos. Dessa maneira determina-se quais são os veículos que comporão uma situação de risco, calculando o tempo necessário para a realização segura da ultrapassagem. 

 \vspace{\onelineskip}

 \noindent
 \textbf{Palavras-chaves}: \imprimirpalavrachaveum, \imprimirpalavrachavedois, \imprimirpalavrachavetres, \imprimirpalavrachavequatro
\end{resumo}
