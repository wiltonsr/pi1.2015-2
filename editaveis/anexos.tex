\begin{anexosenv}

\partanexos

\chapter{RESOLUÇÃO Nº 12/98}

Estabelece os limites de peso e dimensões para veículos que transitem por vias terrestres.

O CONSELHO NACIONAL DE TRÂNSITO - CONTRAN, usando da competência que lhe confere o inciso I, do art. 12, da Lei 9.503 de 23 de setembro de 1997, que instituiu o Código de Trânsito Brasileiro - CTB, e conforme Decreto nº 2.327, de 23 de setembro de 1997, que trata da coordenação do Sistema Nacional de Trânsito;

CONSIDERANDO o art. 99, do Código de Trânsito Brasileiro, que dispõe sobre peso e dimensões; e

CONSIDERANDO a necessidade de estabelecer os limites de pesos e dimensões para a circulação de veículos; resolve:

Art. 1º As dimensões autorizadas para veículos, com ou sem carga, são as seguintes:

I   - largura máxima: 2,60m;

II  - altura máxima: 4,40m;

III - comprimento total:

a) veículos simples: 14,00m;

b) veículos articulados: 18,15m;

c) veículos com reboque: 19,80m.

§ 1º Os limites para o comprimento do balanço traseiro de veículos de transporte de passageiros e de cargas são os seguintes:

I – nos veículos simples de transporte de carga, até 60\% (sessenta por cento) da distância entre os dois eixos, não podendo exceder a 3,50m (três metros e cinqüenta centímetros);

II – nos veículos simples de transporte de passageiros:

a) com motor traseiro: até 62\% (sessenta e dois por cento) da distância entre eixos;

b) com motor central: até 66\% (sessenta e seis por cento) da distância entre eixos;

c) com motor dianteiro: até 71\% (setenta e um por cento) da distância entre eixos.

§ 2º A distância entre eixos, prevista no parágrafo anterior, será medida de centro a centro das rodas dos eixos dos extremos do veículo.

§ 3º Não é permitido o registro e licenciamento de veículos, cujas dimensões excedam às fixadas neste artigo, salvo nova configuração regulamentada por este Conselho.

§ 4º Os veículos em circulação, com dimensões excedentes aos limites fixados neste artigo, registrados e licenciados até 13 de novembro de 1996, poderão circular até seu sucateamento, mediante autorização específica e segundo os critérios abaixo:

I – Para veículos que tenham como dimensões máximas, até 20,00 metros de comprimento; até 2,86 metros de largura, e até 4,40 metros de altura, será concedida Autorização Específica Definitiva, fornecida pela autoridade com circunscrição sobre a via, devidamente visada pelo proprietário do veículo ou seu representante credenciado, podendo circular durante as vinte e quatro horas do dia, com validade até o seu sucateamento, e que conterá os seguintes dados:

a)  nome e endereço do proprietário do veículo;

b) cópia do Certificado de Registro e Licenciamento do Veículo-CRLV;

c) desenho do veículo, suas dimensões e excessos.

II – Para os veículos, cujas dimensões excedam os limites previstos no inciso I, será concedida Autorização Específica Anual, fornecida pela autoridade com circunscrição sobre a via e considerando os limites dessa via, com validade de um ano, renovada até o sucateamento do conjunto veicular, obedecendo os seguintes parâmetros:

a)  volume de tráfego;

b)  traçado da via;

c) projeto do conjunto veicular, indicando dimensão de largura, comprimento e altura, número de eixos, distância entre eles e pesos.

§ 5º De acordo com o art. 101, do Código de Trânsito Brasileiro, as disposições dos parágrafos anteriores, não se aplicam aos veículos especialmente projetados para o transporte de carga indivisível.

% \chapter{Segundo Anexo}

% Texto do segundo anexo.

\end{anexosenv}
