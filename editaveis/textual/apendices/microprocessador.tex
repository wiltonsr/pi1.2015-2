\chapter{MICROPROCESSADOR}

\begin{itemize}

\item \textbf{Características da Arquitetura}

\begin{enumerate}
	\item Arquitetura Load-Store: as instruções somente processarão (soma, subtração, etc) valores que estiverem nos registradores e sempre armazenarão os resultados em algum registrador.
	\item Instruções fixas de 32 bits de largura (com exceção das instruções Thumb compactas de 16 bits) alinhadas em 4 bytes consecutivos da memória, com execução condicional, com poderosas instruções de carga e armazenamento de múltiplos registradores, capacidade de executar operações de deslocamento e na ULA com uma única instrução executada em um ciclo de clock .
	\item Formato de instruções de 3 endereços (isto é, os dois registradores operandos e o registrador de resultado são independentemente especificados)
	\item 15 registradores de 32 bits para uso geral
	\item Manipulação de periféricos de I/O como dispositivos mapeados na memória com suporte à interrupções.
	\item Conjunto de instruções aberto a extensões através de co-processador, incluindo a adição de novos registradores e tipos de dados ao modelo do programador.
	\item Pipeline de 3 a 15 estágios.
	\item Baixo Consumo de energia;
	\item Tamanho do núcleo reduzido.
\end{enumerate}

\end{itemize}