\chapter{Ciclos de Vida}

O número de acidentes nas rodovias preocupa os motoristas e o governo brasileiro. A construção de um dispositivo que auxilie o motorista na hora da ultrapassagem, identificando outros veículos, pode reduzir o número de acidentes e consequentemente o número de mortes e gastos relativos a esses incidentes.

Entretanto, para a construção de um dispositivo que se adeque bem a diversos carros, é necessário um estudo de campo e avaliações sobre o uso de mais componentes eletrônicos nos carros. A poluição visual do motorista é inmensamente prejudicial para suas capacidades de identificação e reflexos. Tendo este ponto inicial para trabalho, é de grande interesse buscar métodos de avaliação do conteúdo disposto na tela do dispositivo.

Neste documento há explicações de como se deu o processo de construção de uma interface que visa o bem estar do usuário  para cumprir com o objetivo principal do dispositivo.

\begin{enumerate}

\item{Ciclos de Vida de desenvolvimento de Interface}

Este ciclo de vida simples incorpora a iteração e encoraja o foco no usuário.

\begin{itemize}
\item Identificar necessidades e definir requisitos

Esta atividade define o processo de identificar as necessidades da aplicação, isto é, elucidar o que a aplicação realmente precisa fazer para cumprir seus objetivos.

\item Design

Nesta fase é construído um modelo de como deve se parecer a solução. São construídos modelos de baixa fidelidade, média fidelidade e alta fidelidade que serão levados à avaliação.

\item Construir uma versão interativa

Após a construção de versões alternativas do protótipo,é construída uma versão que em que o usuário experimentador poderá interagir com o produto.		

\item Avaliar

Fase final do processo Simples em que é avaliado os aspectos de usabilidade do protótipo construído. A partir destas avaliações é possível melhorar os próximos protótipos de acordo com os feedbacks dados pelos avaliadores.

\item \textbf{Ciclo de Vida em Estrela}

O ciclo de vida estrela baseia-se no desenvolvimento como “ondas alternantes”, esta noção de ondas alternantes e devido ao fato de a avaliação esteja presente no começo de cada processo, este ciclo possui  seis principais atividades:

\item Análise de tarefas, usuários e Funções

Aqui é identificado as necessidades e novas oportunidades de negócio.

\item Especificação de Requisitos

Consolida a análise, colocando em pauta os problemas e necessidades da solução.

\item Projeto Conceitual e Especificação de Design

O design de interação é construído de acordo com as especificações.

\item Prototipação

São construídos protótipos para melhor entendimento da solução.

\item Implementação

O sistema interativo final é desenvolvido.

\item Avaliação

Atividade central deste ciclo de vida. Aqui os dados são coletados para futuros redesigns e validação dos requisitos coletados anteriormente. Os relatos dados aqui são valiosos para construção de um sistema mais confiável


\item \textbf{Heurísticas de Usabilidade Nielsen (1993)}
	
Heurísticas de usabilidade são itens de avaliação da usabilidade voltado para ergonomia de um produto [1]. Tem como objetivo evitar erros comuns vindos da interpretação dos requisitos que podem prejudicar a experiência do usuário. Esses itens podem ser avaliados durante o processo de desenvolvimento do produto, através de protótipos.

Para a construção de uma interface de interação com o usuário, pode-se utilizar as dez heurísticas de usabilidade do Jakob Nielsen para guiar a melhor maneira de se implementar a comunicação com o cliente final, as heurísticas são:
\begin{enumerate}
\item Visibilidade de Status do Sistema: Significa que a interface deve sempre informar ao usuário o que está acontecendo, ou seja, todas as ações precisam de feedback instantâneo para orientá-lo.

\item Relacionamento entre a interface do sistema e o mundo real:  Fazer uso de palavras que são do cotidiano do motorista e evitar por completo palavras que definam termos técnicos e complexos para um usuário comum. Toda a comunicação do sistema precisa ser contextualizada ao usuário, e ser coerente com o chamado modelo mental do usuário.

\item Controle e liberdade para o usuário: o sistema deve permitir ao usuário formas de realizar ou desfazer eventos, assim como maneiras fáceis para acessar e sair dos locais que se encontram;

\item Consistência e padrões: Nunca identificar uma mesma ação com ícones ou palavras diferentes. O sistema deve ser construído para tratar ações similares de uma maneira igualitária, facilitando, assim, a identificação das funcionalidades para o usuário.

\item Prevenção de erros: mensagens de confirmação para ações definitivas, assim como evitar que os erros cheguem ao usuário.

\item Reconhecer em vez de relembrar: o sistema deve estar disposto de forma a deixar sempre visível os objetos, ações ou opções a fim de evitar que ele use a memoria durante a interação.

\item Flexibilidade e eficiência de uso: o sistema necessita ser de fácil compreensão para os usuários, independentemente da experiencia na utilização;

\item Estética e design minimalista: para melhor entendimento do usuário, a interação deve mostrar somente o necessário, com diálogos simples;

\item Suporte para o usuário reconhecer, diagnosticar e recuperar erros: os erros ao serem exibidos ao usuário devem ser simples e de fácil entendimento com poucas ações para a sua resolução.

\item Ajuda e documentação: Ajuda e documentação são necessários para sanar dúvidas relativas ao uso do produto mesmo que sua interface seja bem construída. Deve ser facilmente acessível  e conter todas as informações relativas ao dispositivo.
\end{enumerate}

\item \textbf{Planejamento de Avaliações}

O Framework DECIDE é um modelo proposto por Preece et alii (2005, p.368) composto por seis etapas que descrevem um caminho seguro para construir uma avaliação de Interação humano computador, ajudando assim os avaliadores inexperiêntes a planejar e realizar uma avaliação.


\item Determinar os objetivos gerais que a avaliação deverá tratar

Quais os objetivos gerais da avaliação, quem irá realizá-la.
		
\item Explorar perguntas específicas a serem respondidas
		
Considera-se o usuário e suas atividades (Identifica os utilizadores) para então criar perguntas específicas para a avaliação do sistema.

\item Escolher o paradigma e as técnicas de avaliação que responderão as perguntas
	
Escolher os melhores métodos para atingir os objetivos esperados.

\item Identificar questões práticas que devem ser tratadas

Identificar o perfil de usuário e número de usuários que participarão da avaliação

\item Decidir como lidar com questões éticas
	
Certificar que os direitos dos envolvidos nas avaliações sejam respeitados.

\item Avaliar interpretar e apresentar os dados

Definir confiabilidade e validade dos dados coletados. Os testes podem retornar dados

\item \textbf{Métodos de Avaliação}
\begin{enumerate}

\item Método de Avaliação Analítico
Os métodos de avaliação analíticos possuem o objetivo de auxiliar na inspeção ou examinação das características de uma interface de usuário relacionadas à usabilidade. Ou seja, a avaliação analítica é utilizada na busca por problemas de usabilidade em um projeto de interface existente.

Mack e Nielsen (1994) identificam como principais objetivos deste tipo de avaliação:

Identificação dos problemas de usabilidade: identificar, classificar e contar o número de problemas de usabilidade encontrados durante a inspeção;

Seleção dos problemas que devem ser corrigidos: a equipe do projeto deve identificar os problemas e reprojetar a interface, com o intuito de corrigir o maior número possível de problemas, que são priorizados de acordo com sua gravidade e custo associado à correção.

Existem três tipos de conhecimento envolvidos em uma avaliação analítica, são eles:
\begin{enumerate}
\item O conhecimento sobre o domínio: Usado para determinar o que os usuários querem, o que eles precisam, além das frequências e importâncias das tarefas.

\item O conhecimento e a experiência do projeto de interfaces de usuário: são necessários para auxiliar o avaliador na análise dos aspectos de maior importância em um projeto de interfaces, além de apresentar ao avaliador os princípios e diretrizes disponíveis na literatura e ensiná-lo a distinguir quando aplicar e quando ignorar um princípio ou diretriz.

Com base nestes tipos de conhecimento, é possível avaliar os perfis de possíveis avaliadores de acordo com a seguinte escala, em ordem de preferência:

\item Ideal: Especialista “duplo”. Possui experiência nos princípios de usabilidade e nos aspectos relevantes do domínio.

\item Desejável: Especialista em IHC. Conhece o processo de avaliação, bem como os princípios e diretrizes relevantes. Pode aprender o suficiente do domínio para selecionar as tarefas que deverão ser realizadas.

\item Menos desejável: Especialista no domínio. Tem conhecimento do domínio e estuda princípios de interface e o processo para realizar a avaliação.

\item Menos desejável ainda: Membro da equipe de desenvolvimento. Possui dificuldade em deixar de lado seu papel de desenvolvedor e assumir um ponto de vista semelhante ao de um usuário.
\end{enumerate}

\item Avaliação Heurística

O método de avaliação heurística é um método analítico, bastante rápido, que possui como objetivo identificar problemas de usabilidade através de um conjunto de heurísticas ou diretrizes (guidelines) (Nielsen, 1994). Ele se baseia em melhorias práticas definidas por profissionais experientes e especialistas em IHC, ao longo de diversos anos de trabalho nesta área.

A avaliação deve seguir o seguinte procedimento:
Sessões curtas (1 a 2 horas) de avaliação individual, onde cada especialista:
\begin{enumerate}

\item Julga a conformidade da interface com um determinado conjunto de princípios (“heurísticas”) de usabilidade.

\item Anota os problemas encontrados e sua localização.

\item Julga a gravidade destes problemas.

\item Gera um relatório individual com o resultado de sua avaliação e comentários adicionais.

Consolidação da avaliação dos especialistas:

\item Novo julgamento sobre o conjunto global dos problemas encontrados
\item Relatório unificado de problemas de usabilidade

Seleção dos problemas que devem ser corrigidos.
\end{enumerate}
\end{enumerate}

\item \textbf{Termo de Consentimento}

Termos de consentimento são aplicados para que o usuário esteja ciente do teste que está participando. Serve também para documentar sua experiência. As cartas de consentimento utilizadas:
Exemplos de Termo de Consentimento
	Termo de Consentimento - Exemplo 1

\item Carta de Consentimento

Afirmo que sou maior de 17 anos e desejo participar de uma pesquisa conduzida pelos alunos Gustavo Sabino, João Henrique, Igor Ribeiro,Marcelo Herton e Luis Henrique, ambos alunos de Engenharia de Software na Universidade de Brasília.
O propósito da pesquisa é realizar uma Avaliação de Usabilidade de um Protótipo do Comunicador Interveicular Anticolisão  produzido para a disciplina de Projeto Integrador 1. Fique claro de que esta entrevista não é nenhum teste de capacidade.
Também responderei a perguntas abertas sobre o Protótipo apresentado e a questões sobre a experiência de utilizá-lo.
Eu autorizo o uso dos dados obtidos nesta sessão para construção de publicações futuras contanto que minha identidade continue em sigilo. Fui informado(a) que de que este ato é voluntário, não havendo nenhuma obrigação de realizá-lo se assim o quiser.

Brasília,<data> de Novembro de 2015

\end{itemize}

\end{enumerate}