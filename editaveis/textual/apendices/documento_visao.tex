\chapter[Documento de Visão]{Documento de Visão}
A finalidade deste documento é coletar, analisar e definir necessidades e
recursos de nível superior do CIAC. Ele se concentra nos recursos necessários
aos envolvidos e aos usuários-alvos e nas razões que levam a essas necessidades.
Os detalhes de como o CIAC satisfaz essas necessidades são descritos nas
especificações de requisitos.

No Brasil há um grande número de colisões frontais nas rodovias federais devido
a ultrapassagens, assim como pode ser observado nos dados obtidos no site do
IPEA Figura \cite{ipea}.

Nos últimos dez anos, o Brasil registrou aumento de 50,3\% no número de acidentes
em rodovias federais. As mortes cresceram 34,5\% e a quantidade de feridos, 50\%.
Mas, nos últimos quatro anos, esses números vêm reduzindo; de 2010 a 2014, as
mortes diminuíram aproximadamente 4,5\%. As colisões frontais e atropelamentos
são tipos de acidentes que apresentam baixa ocorrência (6,5\% do total em 2014),
mas respondem por quase metade (48,3\%) das mortes nas rodovias federais. Os
automóveis estão envolvidos na maior parte dos acidentes nas rodovias (75,2\%).
Justamente a falta de atenção dos motoristas que faz com que as colisões frontais,
oriundas de ultrapassagens perigosas, representem 34\% das mortes, de acordo
com a Polícia Rodoviária Federal (PFR). Figura \cite{ipea}.
