\chapter[Documento de Arquitetura]{Documento de Arquitetura}


A finalidade do Documento de Arquitetura é definir um modelo arquitetural para ser aplicado ao desenvolvimento do sistema de anticolisões, bem como reunir todas as informações necessárias ao controle das atividades de Arquitetura, oferecendo uma visão macro dos requisitos arquiteturais e não funcionais para suportar.

\section{Representação da Arquitetura}

\subsection{Metas e restrições da Arquitetura}

O sistema anticolisões tem por objetivo ser um sistema  que auxilia a tomada de decisões na hora da ultrapassagem . Por isso, o sistema  deverá ser seguro e sem falhas. Além disso, a arquitetura do sistema deve ser implementada de modo a permitir a evolução e manutenção do sistema. Abaixo, estão listados os requisitos que devem ser considerados para a arquitetura do sistema:

\begin{enumerate}
  \item Segurança: o sistema deve informar a distância dos carros em relação ao motorista, o sistema deve informar a possibilidade de uma ultrapassagem
  \item Disponibilidade: O sistema deverá estar disponível e ativo 24 horas por dia e 7 dias por semana.
  \item Suportabilidade: O sistema deverá funcionar em um volkswagen Gol 1.0 do ano de 2013.
\end{enumerate}
