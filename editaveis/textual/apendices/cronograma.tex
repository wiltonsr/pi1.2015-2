\chapter{Cronogras}


\begin{table}[]
\centering
\caption{Ponto de Controle 1 – PC  1}
\label{custo_equip}
\begin{tabular}{|p{5cm}|p{3cm}|p{5cm}|}
  \hline
  \textbf{Atividade} & \textbf{Data de Entrega} & \textbf{Responsável}\\
  \hline
  Relatório sobre Transponder & 10/08/2015 & Brenno Taylor\\
  \hline
  Relatório sobre Sensores & 10/08/2015 & Carolina Mieldazis\\
  \hline
  Relatório sobre GPS & 10/08/2015 &Renato Carvalho \\
  \hline
  Separar novos grupos & 10/08/2015 & Gustavo Sabino\\
  \hline
  Realizar novas pesquisas & 10/08/2015 & Executada pelo Grupo em geral\\
  \hline
  Escrever Cronograma & 15/09/2015 & Gustavo Sabino\\
  \hline
  Definição do Escopo & 16/09/2015 & Executada pelo Grupo em geral\\
  \hline
  Escrita da EAP & 21/09/2015 & Executada pelo Grupo em geral\\
  \hline
  Escolha das Tecnologias & 21/09/2015 & Executada pelo Grupo em geral\\
  \hline
  Escopo & 23/09/2015 &Gustavo Sabino \\
  \hline
  Escrever Solução & 23/09/2015 &Executada pelo Grupo em geral \\
  \hline
  Refinar Definição do Sistema & 23/09/2015 & Executada pelo Grupo em geral\\
  \hline
  Escolha das Tecnologias & 23/09/2015 & Executada pelo Grupo em geral\\
  \hline
  Relatório de Gerência & 28/09/2015 &Gustavo Sabino \\
  \hline
  Escrever Definições das Ferramentas de Comunicação & 28/09/2015 & Executada pelo Grupo em geral\\
  \hline
  Análise de Tecnologias & 29/09/2015 & Tainara Costa\\
  \hline
  Documento de Visão & 29/09/2015 & Tiago Ribeiro\\
  \hline
  Análise de Tecnologias Concorrentes & 29/09/2015 & Marcelo Herton\\

  \hline
  Definição de Veículos & 29/09/2015 &Luís Henrique \\
  \hline
  Análise sobre Acidentes em Rodovias Brasileiras & 29/09/2015 &Carolina Mieldazis \\
  \hline
  Entrega do Escopo & 30/09/2015 & Carolina Mieldazis\\
  \hline

\end{tabular}
\end{table}

\begin{table}[]
\centering
\caption{Ponto de Controle 2 – PC  2}
\label{custo_equip}
\begin{tabular}{|p{5cm}|p{3cm}|p{5cm}|}
  \hline
  \textbf{Atividade} & \textbf{Data de Entrega} & \textbf{Responsável}\\
  \hline
  Pesquisa de Concorrentes:

  Volvo, Toyota e Ford;

Quais dispositivos essas marcas utilizam nos seus sistemas;

Funcionamento do sistema;

Custo;

Localização no carro.

 & 17/10/2015 & Grupo CATIA/Materias\\
  \hline
  Pesquisa do Transponder:

Justificativa;

Custo;

Datasheet;

Sugestão de localização no veículo;

Funcionamento.

 & 18/10/2015 & Brenno Taylor\\
  \hline
  Pesquisa de Dispositivos:
GPS;

Câmera;

Sensores locais (LADAR e LIDAR);

Sensor de rotação;

Outros (para complementar o CIAC). & 18/10/2015 & Grupo de Eletrônica e Software\\
  \hline
  Pesquisa das Rodovias mais perigosas e dos carros mais envolvidos em acidentes:

Localização da rodovia

Por que é a rodovia mais perigosa?

Formato da rodovia;

Categorias dos carros mais envolvidos em acidentes;

E o porquê desses carros estarem mais frequentemente envolvidos nos acidentes. & 18/10/2015 & Grupo de Energia\\
  \hline
  Revisão do Relatório:

Revisar todo o relatório, exceto o escopo;

Pesquisar modelos de relatórios de PI1 antigos e atuais (para consultas acerca da disposição e conteúdo).
 & 25/10/2015 & Grupo da Redação\\
  \hline
  Escopo & 25/10/2015 & Renato Carvalho e Tainara Costa\\
  \hline

\end{tabular}
\end{table}

\begin{table}[]
\centering
\caption{Ponto de Controle 3 – PC  3}
\label{custo_equip}
\begin{tabular}{|p{5cm}|p{3cm}|p{5cm}|}
  \hline
  \textbf{Atividade} & \textbf{Data de Entrega} & \textbf{Responsável}\\
  \hline
  Reescrever/melhorar o documento de visão do relatório, observando exemplos de outros grupos;
- Mostrar o antes e o depois assim como o exemplo usado pelo grupo para o melhoramento do Doc.

Melhorar requisitos. & 09/11/2015 & Grupo de Software \\
  \hline
  Revisar RADAR/LIDAR com relação ao alcance, adaptar dispositivos ou trocar por outros com maior alcance (maior ou igual a 770 metros);

  Revisar Câmera para adaptação ao uso noturno, caso não der para adaptar, trocar dispositivo câmera & 09/11/2015 & Grupo de Eletrônica \\
  \hline
  Reescrever/melhorar a introdução do relatório observando exemplos de outros grupos.

 Mostrar o antes e o depois para validação dos gerentes. & 09/11/2015 & Grupo de Energia \\
  \hline
  Adaptação do SCRUM para o nosso projeto.

 Mudar o latex no corpo do documento. Metodologia ao invés da teoria terá essa adaptação do scrum ao nosso grupo.
 & 09/11/2015 & Grupo da Redação \\
  \hline
  Definir laser;

Definir servo motor;

Definir fotorreceptor. & 10/11/2015 & Grupo de Eletrônica \\
  \hline
  Validação da pesquisa com a professora Fabiana. & 11/11/2015 & Grupo de Software \\
  \hline
  Validação da pesquisa com o professor Daniel Munoz. & 11/11/2015 & Grupo de Eletrônica \\
  \hline
  Validação do relatório PC2 com a professora Fabiana. & 13/11/2015 & Grupo da Redação \\
  \hline
  Validação da pesquisa de Consumo Energético com a professora Juliana. & 13/11/2015 & Grupo de Energia \\
  \hline
  Definir características funcionais do sistema;

Criar interface;

Escrever US. & 16/11/2015 & Grupo de Software \\
  \hline
  Trabalhar a integração dos dispositivos;

Criar interface. & 16/11/2015 & Grupo de Eletrônica \\
  \hline
  Melhorar CATIA com detalhamento dos dispositivos;

Renderização com outros exemplos de alocação da caixa protetora (basear-se no Gol).

Desenhar logotipo: CIAC. & 16/11/2015 & Grupo CATIA/Materiais \\
  \hline
  Reescrever apêndices;

Escrever Conclusão do trabalho.
 & 16/11/2015 & Grupo da Redação \\
  \hline
  Apresentar um estudo de viabilidade econômica do produto. & 16/11/2015 & Grupo de Energia \\
  \hline
  Fazer e definir cenário modelo com as informações & 16/11/2015  & Grupo de Software \\
  \hline
  Validação final do relatório com todos os professores orientadores. & 22/11/2015 & Renato Carvalho e Tainara Costa \\
  \hline
\end{tabular}
\end{table}
