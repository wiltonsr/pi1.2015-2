\chapter[Requisitos de Software]{Requisitos de Software}

Como tema de investimento do nosso sistema tem-se: Redução dos acidentes em rodovias

Épicos:

\begin{itemize}
  \item Mostrar dados:
  \begin{itemize}
    \item Interface humano computador
    \item Informar o motorista sobre a ultrapassagem
    \item Informar a localização dos carros em 3 km
    \item Verificar se a ultrapassagem é possível
    \item Verificar as distâncias
  \end{itemize}

  \item Enviar/transmitir/receber dados:
  \begin{itemize}
    \item Parser  (objetos coletados(JSON) )
  \end{itemize}
\end{itemize}

\begin{table}[ht]
\caption{User Store - 01}
\centering
\begin{tabular}{|m {15cm} |}
\hline
\textbf{US-01} \\
\hline
Eu como usuário desejo obter informação visual para esclarecer a minha posição
na rodovia. \\
\hline
\end{tabular}
\label{table:us01}
\end{table}

\begin{table}[ht]
\caption{Critérios de Aceitaração para User Store - 01}
\centering
\begin{tabular}{|m {15cm} |}
\hline
\textbf{Critérios de Aceitação} \\
\hline
O display exibe um mapa com a localização do veículo.\\
\hline
O sistema deve mostrar a posição do carro na rodovia em que trafega. \\
\hline
\end{tabular}
\label{table:cus01}
\end{table}


\begin{table}[ht]
\caption{User Store - 02}
\centering
\begin{tabular}{|m {15cm} |}
\hline
\textbf{US-02} \\
\hline
Eu como motorista gostaria de ter minha posição identificada, para conseguir me
orientar melhor para realizar uma ultrapassagem. \\
\hline
\end{tabular}
\label{table:us01}
\end{table}


\begin{table}[ht]
\caption{Critérios de Aceitaração para User Store - 02}
\centering
\begin{tabular}{|m {15cm} |}
\hline
\textbf{Critérios de Aceitação} \\
\hline
O sistema deve mostrar em um visor a posição do carro na rodovia, mostrando a faixa.\\
\hline
O sistema deve identificar em qual local da faixa o veículo está em tempo real.\\
\hline
\end{tabular}
\label{table:cus01}
\end{table}

\begin{table}[ht]
\caption{User Store - 03}
\centering
\begin{tabular}{|m {15cm} |}
\hline
\textbf{US-03} \\
\hline
Eu como motorista, gostaria de saber a distância do carro que está na minha frente
 \\
\hline
\end{tabular}
\label{table:us03}
\end{table}

\begin{table}[ht]
\caption{Critérios de Aceitaração para User Store - 03}
\centering
\begin{tabular}{| m {15cm} |}
\hline
\textbf{Critérios de Aceitação} \\
\hline
O sistema deve informar em seu visor a distância entre o usuário e o carro a sua frente\\
\hline
A distância deve ser calculada em tempo real. \\
\hline
Deve ser exibida de acordo com o Sistema internacional de medidas. (metros) \\
\hline
\end{tabular}
\label{table:cus01}
\end{table}


\begin{table}[ht]
\caption{User Store - 04}
\centering
\begin{tabular}{|m {15cm} |}
\hline
\textbf{US-04} \\
\hline
Eu como motorista gostaria de saber se é seguro trocar de faixa para ultrapassar. \\
\hline
\end{tabular}
\label{table:us03}
\end{table}

\begin{table}[ht]
\caption{Critérios de Aceitaração para User Store - 04}
\centering
\begin{tabular}{| m {15cm} |}
\hline
\textbf{Critérios de Aceitação} \\
\hline
O sistema deve informar em seu visor a distância entre o usuário e o carro a sua frente\\
\hline
Deve haver um local no visor mostrando se é seguro a troca de faixa \\
\hline
\end{tabular}
\label{table:cus01}
\end{table}


\begin{table}[ht]
\caption{User Store - 05}
\centering
\begin{tabular}{|m {15cm} |}
\hline
\textbf{US-05} \\
\hline
Eu como motorista gostaria de receber um sinal sonoro, quando não for possível ultrapassar, para evitar acidentes. \\
\hline
\end{tabular}
\label{table:us03}
\end{table}

\begin{table}[ht]
\caption{Critérios de Aceitaração para User Store - 05}
\centering
\begin{tabular}{| m {15cm} |}
\hline
\textbf{Critérios de Aceitação} \\
\hline
Deve haver um equipamento que emite som\\
\hline
O equipamento deve emitir o som assim quando haver perigo para ultrapassagem \\
\hline
O som deve ser emitido em bip  \\
\hline
O bip deve permanecer caso o condutor tente realizar a ultrapassagem insegura \\
\hline
\end{tabular}
\label{table:cus01}
\end{table}


\begin{table}[ht]
\caption{User Store - 06}
\centering
\begin{tabular}{|m {15cm} |}
\hline
\textbf{US-06} \\
\hline
Eu como motorista gostaria de saber quais os carros estão na pista, para eu poder tomar a decisão de ultrapassar. \\
\hline
\end{tabular}
\label{table:us03}
\end{table}

\begin{table}[ht]
\caption{Critérios de Aceitaração para User Store - 06}
\centering
\begin{tabular}{| m {15cm} |}
\hline
\textbf{Critérios de Aceitação} \\
\hline
Em um display, o sistema deve mostrar os carros que tem condições de influenciar na ultrapassagem\\
\hline
Cada carro deve ter a sua velocidade de acordo com o sistema internacional de medidas indicada no display.\\
\hline
O display exibe a distância entre os dois carros na mesma direção a frente do condutor\\
\hline
\end{tabular}
\label{table:cus01}
\end{table}


\begin{table}[ht]
\caption{User Store - 07}
\centering
\begin{tabular}{|m {15cm} |}
\hline
\textbf{US-07} \\
\hline
Eu como motorista gostaria de saber se há algum carro vindo na direção contrária para realizar a ultrapassagem. \\
\hline
\end{tabular}
\label{table:us03}
\end{table}

\begin{table}[ht]
\caption{Critérios de Aceitaração para User Store - 07}
\centering
\begin{tabular}{| m {15cm} |}
\hline
\textbf{Critérios de Aceitação} \\
\hline
O sistema deve disponibilizar um visor que mostrará o sentido que os carros em seu raio está indo \\
\hline
\end{tabular}
\label{table:cus01}
\end{table}


\begin{table}[ht]
\caption{User Store - 08}
\centering
\begin{tabular}{|m {15cm} |}
\hline
\textbf{US-08} \\
\hline
Eu como motorista gostaria de saber quantos carros estão na direção contrária para realizar a ultrapassagem. \\
\hline
\end{tabular}
\label{table:us03}
\end{table}

\begin{table}[ht]
\caption{Critérios de Aceitaração para User Store - 08}
\centering
\begin{tabular}{| m {15cm} |}
\hline
\textbf{Critérios de Aceitação} \\
\hline
O sistema deve informar em seu visor a quantidade de carros que podem interferir na ultrapassagem e se encontram na direção contrária.\\
\hline
O sistema deve informar a distância entre os carros da direção contrária. \\
\hline
\end{tabular}
\label{table:cus01}
\end{table}
