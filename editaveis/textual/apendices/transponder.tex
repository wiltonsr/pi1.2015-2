\chapter[TRANSPONDER VDL 4000/VTE CNS SYSTEMS]{TRANSPONDER VDL 4000/VTE CNS SYSTEMS}

\begin{enumerate}
\item \textbf{Conceito}
Transponder é a abreviatura para transmiter/responder, ou seja, um sistema
composto de um leitor, que solicita os dados armazenados em um segundo
transmissor, ou seja, um sistema de comunicação em duas vias.

Existem transponders passivos e ativos:

\begin{itemize}
  \item Passivos: Transponders passivos são frequentemente encontrados,
  sendo equipamentos relativamente simples, podendo ser observados na forma
  de etiquetas ou chips, desprovidos de baterias ou fontes próprias de energia
  , são energizados e ativados quando na presença do leitor, por meio da
   emissão de radiofrequência. Frequentemente utilizados no meio
   comercial/industrial, bem como em chaves codificadas.
  \item Ativos: Transponders ativos são em geral mais complexos que os
  modelos passivos, pois envolvem a emissão de dados via radiofrequência
  tanto na requisição de dados bem como no envio da resposta. Com aplicação
  difundida nos campos aeronáutico e naval, servem como base de
  comunicações/identificação a longa distância. Por serem equipamentos ativos,
  necessitam de uma fonte energética constante. Possuem uma taxa de
  transmissão de dados muito superior aos modelos passivos.
\end{itemize}

Para aplicação em nosso caso, abordaremos apenas o conceito de transponders ativos.

\item \textbf{Como Funciona?}
Um transponder ativo funciona à base de emissão de radiofrequência, tanto para
solicitar os dados de outros dispositivos bem como para emitir os dados
armazenados em si. Desta forma, é necessário notar que um transponder não
fornece dados, sejam eles quaisquer sejam, mas se apresenta como uma solução
versátil para comunicação à longa distância, transmitindo dados
armazenados/coletados pelo sistema ao qual está acoplado.

No caso da aplicação aeronáutica, os transponders são utilizados para
transmitir às estações de controle em solo dados como posição, altitude,
velocidade e direção das aeronaves, sendo que esses dados são obtidos
por diferentes sensores.

Como referência, já existe um sistema de transponders homologado
para identificação veicular em rodovias no Brasil, o SINIAV, que
tem o intuito de identificar os veículos que trafegam pelas
rodovias ao passarem por postos de controle espalhados ao longo do
trajeto. Uma aplicação mais rotineira de uma espécie de transponder
é o sistema de pagamento automático de pedágios em estradas sob concessão.

\item \textbf{Aplicação no caso}
Para o caso ao que esta pesquisa se aplica, o modelo de transponder seria
utilizado para transferir entre os veículos os dados obtidos, ou seja,
velocidade e posição, que serão respectivamente coletados pelo velocímetro
e GPS. Desta forma, conseguimos que esses dados, necessários aos cálculos
base do sistema, sejam transmitidos à distância necessária e com alto grau
de confiabilidade, uma vez que, ao utilizar radiação eletromagnética para
transmitir os dados, um transponder não sofre com interferências climáticas,
ambientais e demais, o que o torna uma boa opção para integrar um sistema crítico.

\item \textbf{Modelo escolhido}
Para integrar o projeto, foi escolhido um sistema já desenvolvido para aplicação
veicular, visando tráfego de equipes de solo em aeroportos, sendo ele o seguinte:

CNS Systems – VDL4000/VTE \cite{datasheet_transponder}
O modelo aqui referenciado apresenta compatibilidade de conexão com os demais
componentes via porta RS422. O alcance de um transmissor VHF é delimitado pela
 linha de visão (line of sight), sendo que esta pode ser calculada da seguinte
 forma:

 $ D = \sqrt{12.7 \times A_{m}} $

 Onde D é o referido alcance, e Am é a altura a que o receptor se encontra
 localizado. No caso, o veículo referência possui uma altura de 1464mm, ou
 1,464 metros, ou seja, o alcance com o transmissor posicionado no teto do
 veículo seria de aproximadamente 4,3 Km em terreno plano, mostrando-se
 adequado à aplicação no caso proposto.

 É importante notar que a frequência utilizada é regulamentada pela ANATEL
 (Agência Nacional de Telecomunicações) para transmissão em solo, não havendo
  assim restrições em relação à radiopropagação ou níveis de energia
  eletromagnética.

\end{enumerate}
