\chapter[JUSTIFICATIVA DA POSIÇÃO E ALOCAÇÃO DOS DISPOSITIVOS]{JUSTIFICATIVA DA POSIÇÃO E ALOCAÇÃO DOS DISPOSITIVOS}
\begin{enumerate}
\item \textbf{Câmera}
A localização da câmera será na região inferior do para-brisa, acima do painel. Este lugar foi escolhido por não causar prejuízo de visão ao motorista e por ser o mais indicado para esse tipo de câmera. Esta localização não prejudica o funcionamento do dispositivo.
\item \textbf{Sensores}
Sensores de rotação são implantados no volante. E funcionam conforme a variação angular. Ao virar o volante, para realizar a ultrapassagem, o sensor calcula o ângulo e envia os dados para o microprocessador.
\item \textbf{Caixa Protetora}
Para a alocação da caixa protetora que contém radar, lidar e GPS, é necessário pensar nas melhores possibilidades levando em conta a temperatura, o espaço e o risco de choque. Localizações as quais a ergonomia do usuário não seja prejudicada e nem ofereça riscos danosos ao equipamento.
\item \textbf{Porta-Luvas}
O espaço destinado ao porta-luvas é uma possibilidade para a locação da caixa. Entretanto, é necessário ressaltar que o usuário não terá esse espaço disponível para uso pessoal devido ao tamanho da caixa. Nesse local, a temperatura não é muito alta e a probabilidade de sofrer choques é pequena.
\item \textbf{Embaixo do banco}
A região localizada embaixo do banco do motorista também pode ser utilizada por se tratar de um espaço com risco de choque baixo e por não apresentar altas temperaturas. Não seria indicado o uso desse espaço em relação ao banco do passageiro devido ao uso dos extintores.
\item \textbf{Porta-malas}
O espaço do porta malas é uma possibilidade, entretanto, a diminuição de um espaço tão visado pelos usuários não é indicada. No caso do modelo Gol da Volkswagem o porta malas contém 285 litros e não deve ser reduzido para que não haja desapontamento dos compradores.
\item \textbf{Teto}
Uma possível alocação da caixa no teto do carro também é possível. Entretanto, é necessário que seja entre os bancos traseiros e dianteiros com uma distância segura das cabeças das pessoas. Levando em conta que o modelo permite a presença de cinco pessoas no interior do veículo, essa caixa precisa ser ajustada de modo a ficar em um local seguro para não machucar o usuário em um possível choque. Assim como o porta malas, essa não é uma localização indicada. Porém, é possível.
\item \textbf{Embaixo do banco traseiro}
A região localizada abaixo do banco traseiro é indicada por se tratar de um local com baixo risco de choque, temperatura estável e por não retirar nenhum espaço usado pelo usuário. A caixa ficaria protegida pelo banco e não sofreria prejuízo no funcionamento dos seus dispositivos.
\end{enumerate}
