\chapter{BR-040}

\begin{itemize}
   \item \textbf{Dados Sobre a rodovia}

   Na rodovia BR-040, o maior número de acidentes ocorreu nos cinco quilômetros ao redor do km 121, no trecho localizado da entrada da RJ-103 até a entrada da RJ-071/081, \cite{pvst} na área da Linha Vermelha. Os acidentes no local somaram 285 ocorrências. Os cinco quilômetros em torno do km 744,1, na região da entrada da MG-452 e da entrada da BR-499, em Santos Dumont (MG), foi o trecho com maior número de mortes, com 7 vítimas fatais.
   Na BR-040, entre Rio de Janeiro e Belo Horizonte, os acidentes com mais vítimas e em maior número ocorrem nas proximidades da Avenida Brasil (RJ) e também na região entre Barbacena e Santos Dumont (MG). Os cinco quilômetros em torno do km 121, na chegada à Avenida Brasil, no Rio de Janeiro, teve 650 acidentes em 2012. O trecho com maior número de mortos, 13 no total, fica localizado em torno do km 112, pouco antes do trevo da Avenida Brasil.

   \item \textbf{Veículos envolvidos em acidentes}

   No caso dos acidentes com caminhões, as ocorrências predominam durante a semana, mas são mais letais aos sábados e domingos. De acordo com o estudo, em 2012, 71 mil acidentes envolveram os veículos comerciais, com 4.230 mortos, ou seja, uma média de 11,6 mortes por dia. “O problema dos acidentes e das mortes envolvendo veículos comerciais é extremamente grave. Ele acarreta perdas que afetam negativamente um dos setores essenciais ao desenvolvimento do País. É preciso ter a dimensão do problema para poder atacar as causas. Por isso a importância de trabalhos estatísticos sobre o tema” destaca J. Pedro Corrêa, consultor do Programa Volvo de Segurança no Trânsito (PVST).
   Os acidentes com caminhões acontecem em maior número durante a semana, no período diurno. Já os mais graves ocorrem aos sábados, com 81 mortes a cada mil acidentes, e aos domingos, com 106 mortes a cada mil acidentes. O amanhecer concentra os acidentes de maior letalidade, em especial no horário entre 4 e 5 horas da manhã.
   Em 2012 aconteceram 62.852 acidentes envolvendo caminhões, com 3.682 vítimas fatais, um número médio de 10,1 mortos por dia. \cite{gpt}
   A falta de atenção lidera o número absoluto de acidentes, com 21.860 ocorrências e índice de gravidade 2,4. Já os acidentes com maior letalidade foram ocasionados pela ultrapassagem indevida, com 2.036 acidentes e índice de gravidade 5,5; seguidos pela ingestão de álcool, com 1.286 acidentes e índice de gravidade 4,5; e pela velocidade incompatível, responsável por 5.368 acidentes e com índice médio de gravidade 4,3.

   Em 2012, ocorreram 10.630 acidentes com ônibus nas rodovias federais. O número de vítimas fatais chegou a 764, uma média de 2,1 mortos por dia. Os acidentes com ônibus acontecem em maior número durante a semana, mas são mais letais aos sábados, com 116 mortes por mil acidentes, e aos domingos, com 138 mortes por mil acidentes. O horário das 4 da manhã é crítico, com as ocorrências mais graves.
   A falta de atenção também foi a maior causa dos acidentes com ônibus em 2012, com 4.092 ocorrências e índice de gravidade 2,5. O principal motivo dos acidentes letais foi dormir ao volante, com 198 acidentes e índice de gravidade 7,2. Os acidentes mais graves também foram causados pela ultrapassagem indevida, com 257 acidentes e índice de gravidade 5,9; velocidade incompatível, com 357 acidentes e índice de gravidade 5,1; e pela ingestão de álcool, com 260 acidentes e índice de gravidade de 5,1.

 \end{itemize}
