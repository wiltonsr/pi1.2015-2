\chapter[Introdução]{Introdução}

Não é difícil imaginar que os acidentes que possuem maior número de vítimas fatais são do tipo
colisão frontal mas a realidade assusta, de 2010 a 2014 quase metade dos acidentes com
fatalidades em rodovias federais, um total de 48,3\%, foram fruto desse tipo de colisão e de
atropelamentos. É um tipo de acidente que embora possua baixa ocorrência é extremamente violento. \cite{ipea}

	Ao contrário do que se pensa, a maioria dos acidentes ocorre a luz do dia, com tempo bom.
  De acordo com a Polícia Rodoviária Federal os motoristas envolvidos em acidentes não estão
  cansados, dirigem no máximo uma hora, sendo a imprudência causa
  principal de acidentes. \cite{acidentesDeTransitoNoBrasil}

	Essas informações recebem comprovação nos relatórios do IPEA, de 2008, informando que colisões
   frontais representam 4\% de todos os acidentes embora possua uma parcela 24,6\% do total de
   mortes. 81,75\% dos acidentes desse tipo ocorreram em pistas simples com tráfego nos dois
   sentidos e o principal motivo apontado seriam: ultrapassagem indevida, ultrapassagem de
   veículos lentos ou em congestionamento e falta de visibilidade. \cite{fatoresCondicionantesGravidade}

	Em outro relatório do IPEA, de 2004, curiosamente mantém a mesma porcentagem apresentada no
   relatório recente de 2008 e pode ser acompanhado na Figura \ref{fig:introducao}. \cite{custos_acidentes_transito}

   \begin{figure}[h]
     \centering
     \includegraphics[width=450px, scale=0.5]{figuras/introducao}
     \caption{Tipo versus Gravidade dos acidentes em rodovias federais – 2004}
     \label{fig:introducao}
   \end{figure}


   Mediante análise dos dados apontados foi identificada uma demanda e este relatório
   contempla o desenvolvimento de um projeto preliminar para um sistema que auxilie o
   condutor a realizar ultrapassagens em rodovias simples, de mão dupla, com segurança.
    O sistema emitirá alertas de colisão indicando quando é seguro ultrapassar um veículo
     por meio de uma rede de comunicação inter-veicular. Este sistema não tem como objetivo
      atuar sobre os comandos do veículo. Será de sua responsabilidade apenas informar
      ao usuário se é segura ou não uma manobra.
