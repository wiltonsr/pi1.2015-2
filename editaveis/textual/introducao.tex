\chapter[Introdução]{Introdução}

\section{Objetivo}
Este relatório contempla o desenvolvimento de um projeto preliminar para um sistema que auxilie o condutor nos momentos de
realizar ultrapassagens em rodovias. O sistema emitirá alertas de colisão indicando quando é seguro ultrapassar um veículo
por meio de uma rede de comunicação inter-veicular. Este sistema não tem como objetivo atuar sobre os comandos do veículo.
Será de sua responsabilidade apenas informar ao usuário se é segura ou não uma manobra.

\section{Dados}

\subsection{Acidentes}

Segundo Ipea, nos últimos dez anos, o Brasil registrou aumento de 50,3\% no número de acidentes em rodovias federais. As mortes cresceram 34,5\% e a quantidade de feridos, 50\%. Mas, nos últimos quatro anos, esses números vêm reduzindo; de 2010 a 2014, as mortes diminuíram aproximadamente 4,5\%.
As colisões frontais e atropelamentos são tipos de acidentes que apresentam baixa ocorrência (6,5\% do total em 2014), mas respondem por quase metade (48,3\%) das mortes nas rodovias federais.
Os automóveis estão envolvidos na maior parte dos acidentes nas rodovias (75,2\%). Justamente a falta de atenção dos motoristas que faz com que as colisões frontais, oriundas de ultrapassagens perigosas, representem 34\% das mortes, de acordo com a Polícia Rodoviária Federal (PFR). Só no ano passado, nas estradas brasileiras, foram registrados 153.677 acidentes e 7.466 mortes \cite{ipea}.
