\chapter[Comunicação e Ferramentas]{Comunicação e Ferramentas}

\section{Canais de Comunicação}

\subsection{Slack}
A plataforma de comunicação em grupo Slack foi adotada pelo grupo como meio de comunicação principal.
Sendo centralizada nela as principais informações e debates relevantes ao escopo do projeto.

\subsection{WhatsApp}
O WhatsApp foi escolhido como meio de comunicação secundário enquanto todos os integrantes do grupo
não migrassem para a plataforma principal. Lá se concentraram os esforços de contato inicial para reunir
todos os integrantes da equipe.

\section{Ferramentas}

\subsection{Google Drive}
O Google Drive foi escolhido como ferramenta para armazenamento dos arquivos e artefatos importantes para a
equipe. As entregas parciais de documentos e pesquisas dos integrantes foram todas centralizada nesta plataforma
para efeitos de backup de documentos.

\subsection{Latex}
O \LaTeX\ foi a ferramenta designada para a estruturação e escrita do relatório.

\subsection{Git}
Como ferramenta de versionamento utilizou-se o Git para gerenciar as mudanças e atualizações do conteúdo do slide
por todos os membros da equipe. Essa ferramenta foi escolhida pelo grupo para evitar diferenças e centralização do
conteúdo do relatório. Pois o mesmo foi armazenado em um repositório online para acesso de todos os membros do projeto.
