\chapter[Comunicação e Ferramentas]{Comunicação e Ferramentas}

\section{Canais de Comunicação}

\subsection{Slack}
A plataforma de comunicação em grupo Slack foi adotada pelo grupo como meio de comunicação principal.
Sendo centralizada nela as principais informações e debates relevantes ao escopo do projeto.

\begin{figure}[h]
  \centering
  \includegraphics[width=300px, scale=0.5]{figuras/slack}
  \caption{Plataforma de Comunicação Slack}
  \label{table:slack}
\end{figure}


\subsection{WhatsApp}
O WhatsApp foi escolhido como meio de comunicação secundário enquanto todos os integrantes do grupo
não migrassem para a plataforma principal. Lá se concentraram os esforços de contato inicial para reunir
todos os integrantes da equipe.

\begin{figure}[h]
  \centering
  \includegraphics[width=300px, scale=0.5]{figuras/wpp}
  \caption{Plataforma de Comunicação WhatsApp}
  \label{table:wpp}
\end{figure}

\section{Ferramentas}

\subsection{Google Drive}
O Google Drive foi escolhido como ferramenta para armazenamento dos arquivos e artefatos importantes para a
equipe. As entregas parciais de documentos e pesquisas dos integrantes foram todas centralizada nesta plataforma
para efeitos de backup de documentos.
\begin{figure}[h]
  \centering
  \includegraphics[width=300px, scale=0.5]{figuras/google_drive-logo}
  \caption{Ferramenta de Armazenamento Google Drive}
  \label{table:google_drive-logo}
\end{figure}
\subsection{Latex}
O \LaTeX\ foi a ferramenta designada para a estruturação e escrita do relatório.
\begin{figure}[h]
  \centering
  \includegraphics[width=200px, scale=0.5]{figuras/latex_logo}
  \caption{Ferramenta de Marcação de Texto LaTeX}
  \label{table:latex_logo}
\end{figure}
\subsection{Git}
Como ferramenta de versionamento utilizou-se o Git para gerenciar as mudanças e atualizações do conteúdo do slide
por todos os membros da equipe. Essa ferramenta foi escolhida pelo grupo para evitar diferenças e centralização do
conteúdo do relatório. Pois o mesmo foi armazenado em um repositório online para acesso de todos os membros do projeto.
\begin{figure}[h]
  \centering
  \includegraphics[width=200px, scale=0.5]{figuras/git}
  \caption{Ferramenta de Versionamento GitHub}
  \label{table:git}
\end{figure}
\subsection{Editores de Texto}
\label{sub:Editores de Texto}
Para editar o textos compilados em LaTeX foram utilizados dois editores de texto diferentes:
\begin{itemize}
  \item Sublime text Editor
  \item Atom text Editor
\end{itemize}

\begin{figure}[h]
  \centering
  \includegraphics[width=200px, scale=0.5]{figuras/sublime}
  \caption{Editor de Texto Sublime}
  \label{table:sublime}
\end{figure}
\begin{figure}[h]
  \centering
  \includegraphics[width=200px, scale=0.5]{figuras/atom}
  \caption{Editor de Texto Atom}
  \label{table:atom}
\end{figure}
