\chapter[Documento de Visão]{Documento de Visão}

\section{Introdução}
A finalidade deste documento é coletar, analisar e definir necessidades e recursos de nível superior do <<Nome do Sistema>>. Ele se concentra nos recursos necessários aos envolvidos e aos usuários-alvo e nas razões que levam a essas necessidades. Os detalhes de como o <<Nome do Sistema>> satisfaz essas necessidades são descritos nas especificações de requisitos.

\subsection{Finalidade}
Este documento tem a finalide de  esclarecer e possibilitar uma visão do sistema que será uma informação importante para o entendimentos dos envolvidos no projeto.
\subsection{Escopo}
Este está inserido em um contexto de especificação dos usuários que estarão utilizando o sistema, os interessados no projeto, os requisitos não funcionais, sistemas semelhantes, recursos e características. Bem como a indicação da documentação que será desenvolvida para auxílio dos usuários ao utilizar o sistema.
\subsection{Visão Geral}
Este documento está dividido em doze sessões, descrevendo o Posiconamento do Projeto, Usuários, Visão Geral do Produto, Recursos do Produto, Restrições, Intervalos de Qualidade, Outros Requisitos do Produto, Requisitos de Documentação, Análise dos Softwares Semelhantes, Desenvolver, Comprar ou Customizar? e por fim, Subsistemas que o <<Nome do Sistema>> possui.


\section{Posicionamento}

\subsection{Oportunidade de Negócios}
Dado o grande numero de acidentes que ocorrem no Brasil, percebe-se que com o desenvolvimento de um sistema simples de alerta na ultrapassagem, pode-se obter ganhos ao evitar gastos com as consequencias dos acidentes para todos os brasilideiros usuarios das rodovias para locomocao.

\subsection{Descrição do Problema}
\begin{tabular}{| l |  p{7cm} |}
\hline
O problema da & Grande quantidade de acidentes, cuminando em mortes, que acontece nas rodovias brasileiras devido a execuçao de ultrapassagens.  \\
\hline
Afeta & O povo brasileiro que e usuario das rodovias como meio de transporte \\
\hline
Cujo o impacto e & A Morte de varios brasileiros, bem como a perca de bens materiais no momento do acidente\\
\hline
Uma boa solucao seria & A Implementaçao de um sistema que possibilite a sinalizaçao ao motorista sobre quando e possivel executar a ultrapassagem \\
\hline
\end{tabular}

\subsection{Sentença de Posição do Produto}
\begin{tabular}{| l |  p{7cm} |}
\hline
Para & Brasileiros que utilizam as rodovias como meio de locomoçao de qualquer carro \\
\hline
Que & Evitara os acidentes no momento que os usuarios das rodovias forem executar ultrapassagens \\
\hline
O <<Nome do produto>> & E uma categoria de sistenas de seguranca\\
\hline
Que & Culminara na queda dos indicares de mortalidade devido a acidentes de transito nas rodovias decorrentes de ultrapassagens \\
\hline
Diferente de & Produtos pre instalados em carros adquiridos em concenssonarias e que nao sao fornecidos no Brasil  \\
\hline
Nosso Produto & Pode ser adiquirido por qualquer motorista para qualquer automovel no Brasil\\
\hline
\end{tabular}


\section{Descrições dos Envolvidos e dos Usuários}

\subsection{Demografia dos Mercados}

\subsection{Resumo dos Envolvidos}

\subsection{Resumo dos Usuários}

\subsection{Ambiente do Usuário}

\subsection{Perfis dos Envolvidos}

\section{Visão Geral do Produto}

\subsection{Perspectiva do Produto}

\subsection{Suposições e Dependências}

\subsection{Custos e Preços}

\subsection{Licenciamento e Instalação}

\section{Recursos do Produto}

\subsection{<aFeature>}

\subsection{<anotherFeature>}


\section{Restrições}
Devido a quantidade de acidentes decorridos de ultrapassagens ser maior em rodovias, nosso sistema tera a restriçao de atender apenas os usuarios transitando nestes locais. Alem disso, o trafico de automoveis em todas as direcoes e muito alto, cuminando assim na ineficiencia do sistema ao detectar a todo momento que a ultrapassagem e inadequada.

Dessa maneira, a utilizacao do projeto dentro de cidades nao viavel, possibilitando o uso apenas em rodivias.

\section{Intervalos de Qualidade}

\section{Outros Requisitos do Produto}

\subsection{Padrões Aplicáveis}

\subsection{Requisitos do Sistema}

\subsection{Requisitos de Desempenho }

\subsection{Requisitos Ambientais}

\section{Requisitos de Documentação}
Esta seção descreve a documentação que deverá ser desenvolvida para suportar a implantação bem-sucedida de aplicativos.

\subsection{Manual do Usuário}
O sitema contera dentro da embalagem a ser adquirida um manual de instruçoes de uso do sistema, bem como orientaçoes sobre instalacao e manutencao.

\subsection{Ajuda On-line}
Sera disponibilizada uma documentacao online para auxilio do usuario em qualquer aspecto de uso. Serao disponibilizados os manuais fisicos, esta documentacao de desenvolvimento do projeto como: Documento de Visao, Especificaçao dos Requisitos e Prototipos de Tela

\subsection{Rotulação e Embalagem}

\subsection{Guias de Instalação e de Configuração, e Arquivo Leiame}

\subsection{Escopo}

\section{Análise dos Softwares Semelhantes}

\section{Desenvolver, Comprar ou Customizar?}

\section{Subsistemas}
