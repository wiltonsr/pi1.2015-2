\chapter[Componentes do Produto]{Componentes do Produto}
\section{GPS}
\subsection{História}
	Inicialmente desenvolvido pelo Departamento de Defesa dos Estados Unidos para a aplicações militares, o Sistema de Posicionamento Global foi liberado para uso de civis na década de 80. Sua base são as informações transmitidas por 24 satélites colocados em órbita pelo Governo americano. Assim, cada um desses satélites cobre uma ampla área da nossa superfície terrestre, emitindo sinais para os receptores GPS apontarem nossa localização.
Disposição dos satélites. \cite{Sistema_de_posicionamento_global}

	Os satélites estão dispostos de forma que sempre haja pelo menos quatro deles emitindo dados para um receptor. Além disso, a tecnologia conta com cinco estações de controle dos satélites, instaladas em diferentes partes da Terra. Elas atualizam as posições dos satélites e sincronizam o relógio atômico integrante em cada um deles.
Funcionamento.

Os satélites circulam a Terra duas vezes ao dia em uma órbita bastante precisa e emitem os seus sinais. Ao receberem essas informações, os receptores utilizam o sistema de triangulação para calcular a exata posição do usuário. Basicamente, o navegador compara o horário em que um sinal foi transmitido com o horário em que foi recebido. Esta diferença de tempo permite que o receptor saiba o quão longe estão os satélites naquele momento. Assim, com os cálculos de distância, ele pode determinar a posição exata do usuário. Com os sinais de pelo menos três satélites, o navegador GPS pode calcular a posição 2D do usuário – latitude e longitude – e, ainda, acompanhar o seu movimento. Já com as informações de quatro ou mais satélites, o receptor pode mostrar a sua posição 3D, ou seja, latitude, longitude e altitude. Assim que a posição do usuário é determinada, o navegador pode calcular outras informações, como velocidade, distância do local de destino, entre outros dados.


	Alguns fatores afetam seu posicionamento, são eles:

\begin{enumerate}
  \item Sinal com caminhos múltiplos: isto ocorre quando o sinal é refletido por objetos como prédios altos ou montanhas, antes de alcançarem o receptor. Isto aumenta o tempo que o sinal leva do satélite até o receptor, causando erros;
  \item Erros do relógio do receptor: o relógio interno do receptor não é tão preciso quanto o relógio atômico dos satélites GPS. Assim podem ocorrer pequenos erros na medição do tempo;
  \item  Erros de órbita
  \item Número de satélites visíveis: quanto mais satélites um receptor GPS puder “enxergar” no céu, melhor a precisão;
\end{enumerate}

Para fins de uso veicular o GPS possui duas áreas abrangentes. São elas: localização para empresas de segurança e monitoramento e o uso em automóveis de competição off road. Um terceiro uso vem ganhando notoriedade, o uso de aplicativos que fornecem informações sobre o trânsito, como o Waze.

\subsection{Justificativa da Escolha}

	O objetivo do presente projeto é desenvolver um sistema de emissão de alertas de colisão veicular em rodovias. Para fazer a identificação da localização dos veículos na rodovia será usado o Sistema de Posicionamento Global (GPS). Por ser uma tecnologia já difundida e estudada no mundo inteiro evoluções do sistema sempre ocorrerão a fim de eliminar alguns dos problemas que impedem seu pleno funcionamento. Tal fato garante a confiabilidade e a segurança necessárias à aplicabilidade do projeto.

\subsubsection{Modelo do receptor}
	O Sistema de Emissão de Alertas Anticolisão Veicular será implantado em todos os veículos a partir da data de conclusão e aceitação do projeto. Dois modelos de receptores serão utilizados:

  \begin{enumerate}
    \item Sistema GPS dos veículos que já possuem de fábrica um kit multimídia;
    \item Garmin 3597 LMT;
  \end{enumerate}

\subsection{Especificações Técnicas}

Dimensões da unidade, LxAxP : 13,8 x 7,7 x 1,3 cm (5,4" L x 3,1" A x 0,5" P)

Tamanho do visor, LxA : 4,3" L x 2,9" A (10,9 x 6,5 cm); 5" diagonal (12,7 cm)

Resolução do visor, LxA : 800 x 480 pixels

Tipo de visor : vidro de múltiplos pontos de toque, dupla orientação, TFT colorido WVGA com luz de fundo branca

Peso : 6,8 oz (192 g)

Bateria : íons de lítio recarregáveis

Duração da bateria : Até 2 horas

Conector para headphone/áudio line-out : Não

Receptor de alta sensibilidade : Sim

À prova d'água : Não

Mapas e memória:

Mapas de ruas pré-carregados : Sim

Inclui actualizações vitalícias de mapas : Sim

Garantia de atualização de mapas : Sim

Visão de terreno 3-D : Sim

Visualização de edifícios em 3D (exibe edifícios em 3D) : Sim

Memória interna : Sim

Aceita cartões de dados : sim cartão microSD™ (não incluído)

Paradas/Favoritos/Localizações : 1

Rotas : 100

Funcionalidades e Vantagens:

Avisos por voz (por exemplo: Vire à direita a 150 metros.) : Sim (altifalante interno com suporte)

Fala o nome das ruas (por exemplo: Vire à direita na ELM Street a 150 metros.)

Navegação ativada por voz (opere o dispositivo com comandos de voz) : Sim

Compatível com Tráfego FM : Sim

Atualizaçoes de trânsito vitalícias gratuitas (recebe gratuitamente atualizaçoes vitalícias de dados de trânsito) : Sim

Sugestão de faixa (guia você à faixa apropriada para a navegação) : Sim

Visão de junções (Exibe placas de saídas) : Sim

myTrends™ (prevê rotas com base no comportamento de navegação do usuário) Sim

trafficTrends™ (calcula rotas com base nas previsões de fluxo de tráfego) : Sim

ecoRoute™ (calcula uma rota para economizar combustível) : Comprar em separado

Classifica automaticamente múltiplos destinos (fornece a rota mais direta) : Sim

Rotas a serem evitadas (evitar rodovias, pedágios etc.) : Sim

Garmin nüLink! Serviços: (receba informações precisas sobre viagens em tempo real) : Não

Compatível com XM® nos EUA : Não

Funcionalidades de navegação segura :

Tecnologia sem fio Bluetooth® : Sim

Indicador de limite de velocidade (exibe o limite de velocidade para a maior parte das estradas nos EUA e Europa) : Sim

Onde estou? (encontre hospitais, delegacias de polícia e postos de gasolina e os endereços e interseções mais próximos) : Sim

Navegação pedestre aprimorada (guia conforme anda pela cidade) : Não

Serviços de saídas (avisa serviços próximos da saída na estrada) : Sim

Garmin Lock™ (recurso anti-furto) : Não


Funcionalidades personalizáveis :

Compatível com veículos Garmin Garage™ (baixe ícones com formatos de carro para o seu dispositivo) : Sim

Garmin Garage™ compatível com vozes (baixe vozes personalizadas para o seu dispositivo) : Sim

Tocador MP3 : Não

Tocador de livros em áudio : Não


\section{Sensor de Longa Distância - DML40-2-1111}


\subsection{Justificativa da Escolha}

O objetivo do presente projeto é desenvolver um sistema de emissão de alertas de colisão veicular em rodovias. Para identificar o veículo que está trafegando na mesma via em direção oposta e obter a sua distância e velocidade relativa para concluir se a ultrapassagem é segura e caso contrário informar o condutor do veículo, será utilizado o um sensor de longa distância que tem seu funcionamento baseado na emissão de um laser infravermelho que será refletido pelo outro veículo e lido pelo sensor.
Segundo o Manual do Planejamento de Acessibilidades e Transportes \cite{sensorLaserDML40} a distância mínima necessária, em metros, para se fazer uma ultrapassagem segura é calculada através da seguinte fórmula:

$DU = 7*VT$
\begin{itemize}
  \item DU - Distância de Ultrapassagem.
  \item VT - Velocidade de Tráfego ( ou relativa )
\end{itemize}

Para rodovias brasileiras nas quais a velocidade máxima é de 110 km/h, de acordo com a legislação prevista pelo Código de Trânsito Brasileiro (CTB) \cite{codigoTransitoBrasileiro}, temos que o sensor precisará ter um alcance mínimo de 770 metros.
Dessa forma, o sensor utilizado será o DML-40-2-1111 da empresa SICK Sensor Intelligence que possui um alcance máximo de 1200 metros, o suficiente para a aplicação do projeto, além de ter um erro considerado satisfatório para ser utilizado em um sistema crítico.

\subsection{ESPECIFICAÇÕES DO SENSOR DML-40-2-1111}
De acordo com o Data Sheet \cite{dataSheetDML40} fornecido pelo fabricante, o sensor escolhido obedece as seguintes especificações:
Especificações gerais em um primeiro momento:

\begin{itemize}
	\item Faixa de medição de 0,5 m até 1.200 m com um reflector;
	\item Medição por tempo de voo;
	\item Possui fácil alinhamento;
	\item Parâmetros livremente programáveis;
	\item Saídas RS-422, RS-232, PROFIBUS, analógicas e duas de comutação;
	\item Parâmetro Near field blanking para a operação através de uma janela de proteção;
\end{itemize}

Benefícios:
\begin{itemize}
	\item Intervalos extremamente longos de medição até 1.200 m sobre alvos naturais, oferecem grande flexibilidade nas aplicações em que a distância é fundamental;

	\item Alinhamento visível especial que permite o alinhamento fácil e rápido, mesmo a longas distâncias, oferecendo instalação rápida e de baixo custo;

	\item Design em forma de caixa de metal resistente para operação e livre de problemas em condições ambientais adversas;

	\item Laser de infravermelho para medição e detecção discreta e segura;
	\item Software de fácil utilização com uma interface de configuração rápida e simples que garante otimização de custo;
	\item Interfaces seriais e analógicas, bem como duas saídas de comutação digitais;
\end{itemize}

\section{Transponder}
\label{sec:Transponder}
\subsection{Conceito}
\label{sub:Conceito}

Transponder é a abreviatura para transmiter/responder, ou seja, um sistema composto de um leitor, que solicita os dados armazenados em um segundo transmissor, ou seja, um sistema de comunicação em duas vias.

Existem transponders passivos e ativos:

\begin{itemize}
  \item Passivos: Transponders passivos são frequentemente encontrados, sendo equipamentos relativamente simples, podendo ser observados na forma de etiquetas ou chips, desprovidos de baterias ou fontes próprias de energia, são energizados e ativados quando na presença do leitor, por meio da emissão de radiofrequência. Frequentemente utilizados no meio comercial/industrial, bem como em chaves codificadas.
  \item Ativos: Transponders ativos são em geral mais complexos que os modelos passivos, pois envolvem a emissão de dados via radiofrequência tanto na requisição de dados bem como no envio da resposta. Com aplicação difundida nos campos aeronáutico e naval, servem como base de comunicações/identificação a longa distância. Por serem equipamentos ativos, necessitam de uma fonte energética constante. Possuem uma taxa de transmissão de dados muito superior aos modelos passivos.
\end{itemize}

Para aplicação em nosso caso, abordaremos apenas o conceito de transponders ativos.

\subsection{Principio de Funcionamento}
\label{subs:Principio de Funcionamento}

Um transponder ativo funciona à base de emissão de radiofrequência, tanto para solicitar os dados de outros dispositivos bem como para emitir os dados armazenados em si. Desta forma, é necessário notar que um transponder não fornece dados, sejam eles quaisquer sejam, mas se apresenta como uma solução versátil para comunicação à longa distância, transmitindo dados armazenados/coletados pelo sistema ao qual está acoplado.

  No caso da aplicação aeronáutica, os tranponder são utilizados para transmitir às estações de controle em solo dados como posição, altitude, velocidade e direção das aeronaves, sendo que esses dados são obtidos por diferentes sensores.

  Como referência, já existe um sistema de tranponders homologado para identificação veicular em rodovias no Brasil, o SINIAV, que tem o intuito de identificar os veículos que trafegam pelas rodovias ao passarem por postos de controle espelhados ao longo do trajeto. Uma aplicação mais rotineira de uma espécie de transponder é o sistema de pagamento automático de pedágios em estradas sob concessão.

\subsection{Aplicação do Transponder}
\label{subs:Aplicação do Transponder}
Para o caso ao que esta pesquisa se aplica, o modelo de transponder seria utilizado para transferir entre os veículos os dados obtidos, ou seja, velocidade e posição, que serão respectivamente coletados pelo velocímetro e GPS. Desta forma, conseguimos que esses dados, necessários aos cálculos base do sistema, sejam transmitidos à distância necessária e com alto grau de confiabilidade, uma vez que, ao utilizar radiação eletromagnética para transmitir os dados, um transponder não sofre com interferências climáticas, ambientais e demais, o que o torna uma boa opção para integrar um sistema crítico.

\subsection{Modelo escolhido}
\label{subs:Modelo escolhido}
Para integrar o projeto, são sugeridas duas opções, a primeira sendo um sistema já desenvolvido para aplicação veicular, visando tráfego de equipes de solo em aeroportos, sendo ele o seguinte:

CNS Systems – VDL4000/VTE \cite{transponderVDL4000}

Entretanto, existe o risco de que o alcance fornecido pelo equipamento sugerido, não especificado no datasheet (necessário contatar o fornecedor), não seja o suficiente para suprir as demandas do projeto. Neste caso, não sendo possível a utilização do primeiro modelo, a equipe sugere como base um modelo de transponder aéreo, o qual seria modificado para se adaptar às necessidades da operação em solo. O modelo sugerido seria o seguinte:

CNS Systems – VDL4000/GA \cite{transponderVDLGA}
