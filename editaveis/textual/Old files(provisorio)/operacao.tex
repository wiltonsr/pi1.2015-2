\chapter[Condições de Operação]{Condições de Operação}

\section{Tipos de veículos}
Para que o sistema funcione corretamente na identificação de carros a longa distância será utilizado o transponder. O mesmo necessita que todos os automóveis tenham-no instalado e que funcionem corretamente. Esta condição se aplica a todos os veículos automotores terrestres, ou seja, aqueles que são capazes de se deslocar pelo impulso gerado por seu motor próprio, tais como: carros, caminhonetes, ônibus, caminhões, tratores, motocicletas e semelhantes \cite{egov}. O relatório de acidentes de trânsito brasileiro, mostra que o maior número de ocorrências de vítimas com lesões graves ou morte é com automóveis particulares \cite{ipea2}.

\section{Alterações realizadas por terceiros}
O projeto propõe um sistema completo para realizar a comunicação interveicular, assim como a localização geográfica através do GPS, e o devido processamento das informações transmitidas. Qualquer tipo de alteração realizada neste sistema pelo usuário final do produto pode ocasionar mal funcionamento: erro na transmissão dos dados, erro de leitura dos sensores utilizados, mal processamento dos dados. Assim, nestes casos o projeto, por tratar de um sistema crítico não garante o seu correto funcionamento em casos de modificação em algum componente, seja físico ou digital.
